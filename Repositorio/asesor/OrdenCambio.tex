\documentclass[10pt,letterpaper,twoside]{article}

\newcommand{\empresa}[0]{\textbf{[[empresa-nombre-largo]]}}
\newcommand{\diahora}[0]{\textbf{[[fecha-actual]]}, a horas [[hora-actual]] }
\newcommand{\lugar}[0]{\textbf{[[lugar]]}}

\newcommand{\pp}[0]{\textbf{[[primer-puntaje]]}}
\newcommand{\sep}[0]{\textbf{[[segundo-puntaje]]}}
\newcommand{\tp}[0]{\textbf{[[tercer-puntaje]]}}
\newcommand{\cp}[0]{\textbf{[[cuarto-puntaje]]}}
\newcommand{\qp}[0]{\textbf{[[quinto-puntaje]]}}
\newcommand{\ssp}[0]{\textbf{[[sexto-puntaje]]}}
\newcommand{\sssp}[0]{\textbf{[[septimo-puntaje]]}}

%\newcommand{\observacionesDetalle}[1]{\item #1}
\newcommand{\observaciones}[0]{\observacionesDetalle{[[obs-detalle-item]]}}  

\usepackage{t1enc}
\usepackage[latin1]{inputenc}
\usepackage[activeacute, spanish]{babel}
\usepackage{graphics}

%\pagestyle{myheadings}
\markboth{TIS}{Ma. Leticia Blanco Coca}

\begin{document}

\title{Orden de Cambio}
\author{Ma. Leticia Blanco Coca}
\maketitle


TIS ha revisado la propuesta que su empresa a entregado y se ha puntuado de la siguiente manera:

\begin{tabular}{|l|l|l|}
\hline \textbf{Descripci'on}& \textbf{Puntaje}&\textbf{Puntaje} \\
& \textbf{Referencial}&\textbf{Obtenido} \\
\hline Cumplimiento de especificaciones del proponente       & 15 puntos & \pp \\
\hline Claridad en la organizaci'on de la empresa proponente & 10 puntos & \sep \\
\hline Cumplimiento de especificaciones t'ecnicas            & 30 puntos & \tp \\ 
\hline Claridad en el proceso de desarrollo                  & 10 puntos & \cp \\
\hline Plazo de ejecuci'on                                   & 10 puntos & \qp \\
\hline Precio total                                          & 15 puntos & \ssp \\
\hline Uso de herramientas en el proceso de desarrollo       & 10 puntos & \sssp \\
\hline
\end{tabular}

TIS despu'es de revisar la propuesta de su empresa \empresa, tiene las siguientes observaciones:

\begin{enumerate}
\observaciones
\end{enumerate}

Esta adenda de correcci'on debe ser entregada por correo electr'onico antes de la firma del contrato, misma que debe hacerse llegar al correo leticia@memi.umss.edu.bo.


Paralelamnete se solicita, indicar el d'ia de su preferencia para realizar revisiones, puesta
en marcha  y 
seguimiento de su propuesta de desarrollo en el tiempo que dure el contrato con TIS.


As'imismo, recordar que para el d'ia de la firma del contrato que se realizar'a el d'ia \diahora en \lugar; se requiere de una copia f'isica
 de la 
\texttt{Boleta de Garant'ia}, emitida a favor de \texttt{TIS} por parte de \empresa.
\end{document}
